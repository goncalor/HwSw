\paragraph{} Nesta seccção são apresentados os resultados temporais e de compressão de vários ficheiros de exemplo. Nos resultados temporais compara-se o desempenho da implementação puramente em software com a solução com aceleração por hardware.

\subsection{Temporais}

  \paragraph{} Na \autoref{tab:time_software} estão patentes os tempos de execução da implementação totalmente em software do algoritmo de compressão. Pode-se notar que existem dois passos que correm em tempo praticamente constante: construir a árvore e transformar a árvore numa tabela. Embora o tempo destes passos não seja constante, é proporcional ao número de caracteres distintos existentes no ficheiro; e consequentemente é limitado superiormente. Este é o motivo pelo qual as operações sobre a árvore para o ficheiro \texttt{pdf} demoram menos tempo do que para \texttt{teste}: o ficheiro \texttt{teste} tem 25 caracteres distintos enquanto que \texttt{pdf} tem apenas 14, logo as operações sobre a árvore demoram mais tempo para o primeiro.


  \begin{table}[H]
    \caption{Tempo de execução de cada parte do algoritmo, por software}

    \centerline
    {
      \begin{tabular}{|c|c|c|c|c|r|}
        \hline
        ficheiro    &
        estatísticas    &
        construir árvore &
        árvore $\rightarrow$ tabela
        & codificação     &
        \multicolumn{1}{c|}{tamanho} \\
        \hline
        \hline
        tmp         & \ 1 ms          & 0 ms           & 0 ms         & \ 0 ms          & 14 B         \\ \hline
        teste       & \ 1 ms          & 5 ms           & 1 ms         & \ 0 ms          & 28 B         \\ \hline
        pdf         & \ 1 ms          & 3 ms           & 0 ms         & \ 1 ms          & 44 B         \\ \hline
        read\_me    & 21 ms           & 12 ms          & 1 ms         & 43 ms           & 2981 B       \\ \hline
        BIG\_READ   & 1,28 min.       & 32 ms          & 3 ms         & 2,66 min.       & 10.4 MB      \\
        \hline
      \end{tabular}
    }

    \label{tab:time_software}
  \end{table}


  O acelerador está descrito na Secção~\ref{sec:accelerator}. O tempo de execução da solução em software é comparado ao da solução hardware/software na \autoref{tab:time_hardware}.


  \begin{table}[H]
    \centering
    \caption{Comparação das contagens \\em software e em hardware/software}

    \begin{tabular}{|c|c|c|}
      \hline
      ficheiro   & software   & hardware/software \\ \hline \hline
      tmp        & \ 1 ms     & 1 ms         \\ \hline
      teste      & \ 1 ms     & 1 ms         \\ \hline
      pdf        & \ 1 ms     & 1 ms         \\ \hline
      read\_me   & 21 ms      & ?          \\ \hline
      BIG\_READ  & 1.28 min.  & ?         \\
      \hline
    \end{tabular}
    \label{tab:time_hardware}
  \end{table}

  % comentar a aceleração que foi conseguida
