\paragraph{} Nesta seccção são apresentados os resultados temporais e de compressão de vários ficheiros de exemplo. Nos resultados temporais compara-se o desempenho da implementação puramente em software com a solução com aceleração por hardware.

\subsection{Temporais}

  \paragraph{} Na \autoref{tab:time_software} estão patentes os tempos de execução da implementação totalmente em software do algoritmo de compressão. Pode-se notar que existem dois passos que correm em tempo praticamente constante: construir a árvore e transformar a árvore numa tabela. Embora o tempo destes passos não seja constante, é proporcional ao número de caracteres distintos existentes no ficheiro; e consequentemente é limitado superiormente. Este é o motivo pelo qual as operações sobre a árvore para o ficheiro \texttt{pdf} demoram menos tempo do que para \texttt{teste}: o ficheiro \texttt{teste} tem 25 caracteres distintos enquanto que \texttt{pdf} tem apenas 14, logo as operações sobre a árvore demoram mais tempo para o primeiro.

  O acelerador é descrito na Secção~\ref{sec:accelerator}. O tempo de execução da solução em software é comparado ao da solução hardware/software na \autoref{tab:time_hardware}. Através desta tabela notamos que a aceleração conseguida é significativa: a solução por hardware é cerca de 4 vezes mais rápida. Para os ficheiros mais pequenos não houve aceleração, ou a mesma foi escondida por os tempos medidos não serem mais precisos.


  \begin{table}[h]
    \caption{Tempo de execução de cada parte do algoritmo, por software}
    \centerline
    {
      \begin{tabular}{|c|r|c|c|r|r|}
        \hline
        \textbf{Ficheiro}                             &
        \multicolumn{1}{c|}{\textbf{Estatísticas}}    &
        \textbf{Construir Árvore}                     &
        \textbf{Árvore $\rightarrow$ Tabela}          &
        \multicolumn{1}{c|}{\textbf{Codificação}}     &
        \multicolumn{1}{c|}{\textbf{Tamanho}} \\
        \hline
        \hline
        tmp       & \ 1 ms    & \ 0 ms & 0 ms & \ 0 ms    & 14 B    \\ \hline
        teste     & \ 1 ms    & \ 5 ms & 1 ms & \ 0 ms    & 28 B    \\ \hline
        pdf       & \ 1 ms    & \ 3 ms & 0 ms & \ 1 ms    & 44 B    \\ \hline
        read\_me  & 21 ms     & 12 ms  & 1 ms & 43 ms     & 2981 B  \\ \hline
        alice     & 1248 ms   & 32 ms  & 3 ms & 2611 ms   & 174 KB  \\ \hline
        BIG\_READ & 76584 ms  & 32 ms  & 3 ms & 159882 ms & 10,4 MB \\
        \hline
      \end{tabular}
    }

    \label{tab:time_software}
  \end{table}



  \begin{table}[h]
    \centering
    \caption{Comparação das contagens em software e em \\hardware/software e respectivo \textit{speedup}}

    \begin{tabular}{|c|r|r|r|}
      \hline
      \textbf{Ficheiro}                                 &
      \multicolumn{1}{c|}{\textbf{Software}}            &
      \multicolumn{1}{c|}{\textbf{Hardware/Software}}   &
      \multicolumn{1}{c|}{\textbf{Speedup}}       \\ \hline \hline
      tmp        & \ 1 ms     & 1 ms     & 1,00   \\ \hline
      teste      & \ 1 ms     & 1 ms     & 1,00   \\ \hline
      pdf        & \ 1 ms     & 1 ms     & 1,00   \\ \hline
      read\_me   & 21 ms      & 6 ms     & 3,50   \\ \hline
      alice      & 1248 ms    & 314 ms   & 3,97   \\ \hline
      BIG\_READ  & 76584 ms   & 19231 ms & 3,98   \\
      \hline
    \end{tabular}
    \label{tab:time_hardware}
  \end{table}


  \subsection{Compressão}

  A \autoref{tab:compression} apresenta o tamanho dos ficheiros de teste após compressão, bem como os seus tamanhos originais.

  \begin{table}[h]
    \centering
    \caption{Compressões geradas para os ficheiros de teste}
    \begin{tabular}{|c|r|r|r|}
        \hline
        \textbf{Ficheiro}                                 &
        \multicolumn{1}{c|}{\textbf{Original}}            &
        \multicolumn{1}{c|}{\textbf{Comprimido}}          &
        \multicolumn{1}{c|}{\textbf{Rácio}}\\ \hline \hline
        tmp        &  13,0  B    &  11,00 B          & 85,0\%     \\ \hline
        teste      &  27,0  B    &  49,00 B          & 181,5\%    \\ \hline
        pdf        &  43,0  B    &  39,00 B          & 90,7\%     \\ \hline
        read\_me   &  3,0  KB    &  1,62 KB          & 54,0\%     \\ \hline
        alice      &  174,0 KB   &  103,79 KB        & 59,6\%     \\ \hline
        BIG\_READ  &  10,4 MB    &  5,99 MB          & 57,6\%     \\
        \hline
    \end{tabular}
    \label{tab:compression}
  \end{table}

  A partir destes resultados pode-se afirmar que todos os ficheiros apresentam uma boa razão de compressão, à excepção do ficheiro \texttt{teste}. Este é um caso extremo do algoritmo de Huffman: neste caso o ficheiro possui uma variedade significativa de caracteres, o que aliado ao seu tamanho reduzido torna difícil a sua compressão visto que a árvore gerada é relativamente grande e, consequentemente, o cabeçalho gerado para o ficheiro comprimido representa um grande \textit{overhead}. Por este motivo, o algoritmo, em vez de comprimir o ficheiro expandiu-o em 81,5\%.
