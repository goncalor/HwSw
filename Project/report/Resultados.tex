\paragraph{} Nesta seccção são apresentados os resultados temporais e de compressão de vários ficheiros de exemplo. Nos resultados temporais compara-se o desempenho da implementação puramente em software com a solução com aceleração por hardware.

\subsection{Temporais}

  \paragraph{} Na \autoref{tab:time_software} estão patentes os tempos de execução da implementação totalmente em software do algoritmo de compressão. Pode-se notar que existem dois passos que correm em tempo praticamente constante: construir a árvore e transformar a árvore numa tabela. Embora o tempo destes passos não seja constante, é proporcional ao número de caracteres distintos existentes no ficheiro; e consequentemente é limitado superiormente. Este é o motivo pelo qual as operações sobre a árvore para o ficheiro \texttt{pdf} demoram menos tempo do que para \texttt{teste}: o ficheiro \texttt{teste} tem 25 caracteres distintos enquanto que \texttt{pdf} tem apenas 14, logo as operações sobre a árvore demoram mais tempo para o primeiro.


  \begin{table}[H]
    \caption{Tempo de execução de cada parte do algoritmo, por software}

    \centerline
    {
      \begin{tabular}{|c|c|c|c|c|r|}
        \hline
        \textbf{Ficheiro}                    &
        \textbf{Estatísticas}                &
        \textbf{Construir Árvore}            &
        \textbf{Árvore $\rightarrow$ Tabela} &
        \textbf{Codificação}                 &
        \multicolumn{1}{c|}{\textbf{Tamanho}} \\
        \hline
        \hline
        tmp       & \ 1 ms    & \ 0 ms & 0 ms & \ 0 ms    & 14 B    \\ \hline
        teste     & \ 1 ms    & \ 5 ms & 1 ms & \ 0 ms    & 28 B    \\ \hline
        pdf       & \ 1 ms    & \ 3 ms & 0 ms & \ 1 ms    & 44 B    \\ \hline
        read\_me  & 21 ms     & 12 ms  & 1 ms & 43 ms     & 2981 B  \\ \hline
        BIG\_READ & 1,28 min. & 32 ms  & 3 ms & 2,66 min. & 10.4 MB \\
        \hline
      \end{tabular}
    }

    \label{tab:time_software}
  \end{table}


  O acelerador está descrito na Secção~\ref{sec:accelerator}. O tempo de execução da solução em software é comparado ao da solução hardware/software na \autoref{tab:time_hardware}.


  \begin{table}[H]
    \centering
    \caption{Comparação das contagens em software e em \\hardware/software tal como o seu Speed Up}

    \begin{tabular}{|c|c|c|c|}
      \hline
      \textbf{Ficheiro}            &
      \textbf{Software}            &
      \textbf{Hardware/Software}   &
      \textbf{Speed Up}\\ \hline \hline
      tmp        & \ 1 ms     & 1 ms              & 1,00         \\ \hline
      teste      & \ 1 ms     & 1 ms              & 1,00         \\ \hline
      pdf        & \ 1 ms     & 1 ms              & 1,00         \\ \hline
      read\_me   & 21 ms      & 6 ms              & 3,50         \\ \hline
      BIG\_READ  & 1,28 min.  & 19231 ms          & 901,45       \\
      \hline
    \end{tabular}
    \label{tab:time_hardware}
  \end{table}

  A partir dos resultados da \autoref{tab:time_hardware} não se nota uma grande aceleração com ficheiros de dimensão pequena mas com ficheiros de grandes dimensões, tal como \texttt{read\_me} e \texttt{BIG\_READ}, nota-se uma aceleração significativa na contagem de caracteres. Este facto provém da leitura do ficheiro ter complexidade de $O(n)$ - linear.

  \subsection{Compressão}

  A \autoref{tab:compression} apresenta o tamanho dos ficheiros teste comprimidos tal como os seus valores originais.

  \begin{table}[H]
    \centering
    \caption{Compressões geradas para os ficheiros de teste}
    \begin{tabular}{|c|r|r|r|}
        \hline
        \textbf{Ficheiro}                                 &
        \multicolumn{1}{c|}{\textbf{Original}}            &
        \multicolumn{1}{c|}{\textbf{Comprimido}}          &
        \multicolumn{1}{c|}{\textbf{Rácio}}\\ \hline \hline
        tmp        &  13  B     &  11 B             & 85,0\%     \\ \hline
        teste      &  27  B     &  49 B             & 181,5\%    \\ \hline
        pdf        &  43  B     &  39 B             & 90,7\%     \\ \hline
        read\_me   &  3  KB     &  1,62 KB          & 54,0\%     \\ \hline
        BIG\_READ  &  10,4 MB   &  5,99 MB          & 57,6\%     \\
        \hline
    \end{tabular}
    \label{tab:compression}
  \end{table}

  A partir destes resultados pode-se afirmar que todos os ficheiros apresentam uma boa margem de compressão à excepção do ficheiro \texttt{teste}. Este é o caso extremo do algoritmo de Huffman, neste caso o ficheiro possui uma grande variedade de caracteres o que torna difícil a sua compressão pois a árvore gerada é muito equilibrada e muito profunda, ou seja, a representação dos caracteres a partir da árvore é maior do que a representação original ASCII. Por este motivo, o algoritmo, em vez de comprimir o ficheiro expandiu-o em 81,5\%.
