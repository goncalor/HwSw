\paragraph{} Nesta secção é descrita a implementação por software.

\subsection{Estatísticas do Ficheiro}

\paragraph{} Para obter a frequência absoluta de cada carácter do ficheiro é feito um varrimento do ficheiro e incrementado um contador relativo ao número de ocorrências desse carácter. Isto é feito até que seja encontrado o carácter terminador. Esse carácter está definido no programa como \texttt{FILE\_END\_CODE}. Ao longo desta parte do trabalho tem-se considerado como terminador o carácter \texttt{0x00}.

A razão pela qual é preciso escolher um carácter como terminador prende-se com o facto de não termos um sistema operativo a correr no MicroBlaze. Caso tivesse um sistema operativo, ele podia notificar o programa quando não houvesse mais conteúdo para ler.

A complexidade desta fase do algoritmo é linear com o tamanho do ficheiro. Antes de se passar à construção da árvore, é construído um vector com pares (carácter, contagem), que não inclui caracteres com frequência absoluta nula. Na construção da árvore são portanto considerados apenas caracteres com pelo menos uma ocorrência. Isto resulta numa árvore mais pequena (menor \textit{overhead} no ficheiro comprimido) e torna a construção da árvore mais rápida.

\subsection{Árvore}

\paragraph{} Tal como visto na Secção~\ref{sec:theory} uma operação muito frequente durante a construção da árvore de Huffman é encontrar as duas raízes que em cada momento têm menor frequência absoluta. De forma a acelerar estas procuras, fez-se a implementação da árvore recorrendo a um acervo (\textit{heap}) que contém apontadores para os nós da árvore. O acervo é construído com complexidade $O(n)$, permite remoção do elemento com menor frequência em $O(log\ n)$ e inserção também $O(n)$, em que $n$ é o número de elementos, sendo que neste caso temos $n \leq 256$.

Por cada dois elementos (nós) que são removidos do acervo é adicionado um novo com frequência igual à soma das frequências dos que foram removidos. Os nós removidos são depois ``pendurados'' debaixo do novo nó. Assim sendo, cada elemento do acervo é na verdade uma árvore. Quando o acervo tiver apenas um elemento temos certeza de que esse constitui a árvore de Huffman. Note-se que por cada dois elementos que são removidos é adicionado apenas um, pelo que construir a árvore de Huffman demora, no máximo, 255 passos.

\subsection{Ficheiro de Saída}

\paragraph{} Percorrendo a árvore de Huffman é possível obter o código que corresponde a cada carácter. De forma a optimizar o processo de codificação do ficheiro, começa-se por percorrer a árvore com o objectivo de construir uma tabela indexada pelos caracteres, e que em cada posição contem um par (código, comprimento), em que \emph{comprimento} é o número de bits do código. Recorrendo a esta tabela é agora possível aceder ao código de cada carácter em tempo constante.

Para codificar o ficheiro resta agora substituir os caracteres do ficheiro pelos seus códigos. Esta operação não é assim tão trivial visto que a unidade mínima de memória que é possível endereçar é o byte. Por este motivo é necessário codificar vários caracteres para construir cada byte do ficheiro de saída. Isto implica uma série de operações lógicas (\textit{shifts} e \textit{ORs}) e aritméticas que demoram uma quantidade apreciável de tempo, como se pode notar na secção~\nameref{sec:results}.
