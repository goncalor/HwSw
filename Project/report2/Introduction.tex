Na 1ª parte a aplicação de codificação de Huffman foi acelerada através da adição de \textit{hardware} na secção da contagem de caracteres. Nesta 2ª parte do projecto pretende-se melhorar o sistema da 1ª parte de forma a acelerar ainda mais a aplicação. Para isso é introduzido mais um factor de aceleração: a paralelização do processamento de dados, através de 4 processadores \textit{softcore} \textit{MicroBlaze}.

Da análise feita na 1ª parte observou-se que embora a aplicação tenha sido acelerada, esta não apresentava um \textit{speedup} global significativo (o \textit{speedup} das estatísticas era cerca de 400\%, mas o global era apenas de 30\%). A área escolhida na aceleração da 1ª parte foi a contagem de caracteres mas esta não era a área que despendia mais tempo de processamento, essa área pertencia à escrita do ficheiro comprimido. Nesta 2º parte para além de paralelizar as contagens com o acelerador de \textit{hardware} também foi paralelizada a secção de escrita do ficheiro comprimido.

Com este objectivo em mente, apresenta-se neste documento, a configuração do sistema usado de forma a melhorar o desempenho da aplicação e as modificações feitas ao algoritmo que permitem que a aplicação execute correctamente no sistema actualizado.