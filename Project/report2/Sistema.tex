O sistema opera a 75~MHz com 4 \textit{MicroBlazes} com a seguinte configuração:
o processador mestre possui uma memória local de 64~kB, 2 \texttt{AXI~Timers} de 32 bits ligados em cascata --- que resulta num \textit{timer} de 64 bits --- e uma \textit{cache} de 8~kB de dados. Os restantes processadores, escravos, possuem uma memória local de 16~kB com a mesma configuração de \textit{cache} (ambas as \textit{caches} apresentam uma política \textit{write-through}). São usadas duas memórias internas, uma de 4~kB e outra de 16~kB. A primeira está ligada ao \textit{bus} \texttt{AXI Lite} --- para não ter cache; a segunda ao \textit{bus} \texttt{AXI} --- de forma a tomar partido da cache --- é usada para guardar a tabela de conversões, de carácter para representação de Huffman.

Em suma, o sistema é constituído pelos seguintes componentes:

\begin{itemize}
	\item 1 \textit{MicroBlaze} mestre
    \begin{itemize}
		\item 64~kB de memória local
        \item 8~kB de \textit{cache} de dados; 64~B de \textit{cache} de instruções
	\end{itemize}
    \item 3 \textit{MicroBlazes} escravos
    \begin{itemize}
		\item 16~kB de memória local
        \item 8~kB de \textit{cache} de dados; 64~B de \textit{cache} de instruções
	\end{itemize}
    \item 1 memória interna de 4~kB, sem \textit{cache}
    \item 1 memória interna de 16~kB, com \textit{cache}
    \item 1 memória externa de 128~MB, com \textit{cache}
    \item 1 temporizador AXI de 64 bits (2 de 32, em cascata)
    \item 1 módulo UART
\end{itemize}


A memória externa está ligada através do \textit{bus} AXI4, de maneira a tomar partido da \textit{cache} de dados e instruções. A memória interna que não está ligada à cache é usada da seguinte forma:

\begin{itemize}
\item Primeiros 4 bytes para sincronizar os vários processadores;
\item Endereços de 0x4 a 0x403 são usados pelo processador 1 para partilhar contagens com o processador 0;
\item Endereços de 0x404 a 0x803 são usados pelo processador 2 para partilhar contagens com o processador 0;
\item Endereços de 0x804 a 0xC03 são usados pelo processador 3 para partilhar contagens com o processador 2;
\item As restantes posições não são utilizadas.
\end{itemize}


O programa usado no processador mestre usa cerca de 54~kB e os restantes cerca de 9~kB. Ao processador mestre foi atribuída uma \textit{stack} de 10~kB, pois a pesquisa na árvore é feita recursivamente. Como no pior caso a árvore tem 256 níveis podem ser ocupados aproximadamente $256 \times (4 + 4 + 4) = 3072$ B em \textit{stack}, visto que cada chamada recursiva usa dois ponteiros mais um \texttt{char} e \texttt{short} --- assumindo que são usados 4 bytes devido a alinhamentos. Contabilizando outros dados que são colocados na \textit{stack} por cada chamada, estima-se que o tamanho da \textit{stack} possa chegar a cerca de 5~kB. Para ter margem de segurança definiu-se no \textit{linker script} uma \textit{stack} com o número redondo de 10~kB. Os restantes processadores usam 1~kB de \textit{stack} e de \textit{heap}, visto que o processamento feito não necessita de uma grande quantidade de memória.

Não foi usada qualquer optimização por parte do compilador (\texttt{-O0}).

Alguns valores relativos à ocupação do sistema (valor absoluto e percentagem de ocupação) são:


\begin{verbatim}
            Slice registers               8 652       15 %
            Slice LUTs                   11 981       43 %
               used as Memory             1 855
            occupied Slices               4 355       63 %
            LUT Flip Flop pairs used     13 690
            RAMB16BWERs                      86       74 %
            RAMB8BWERs                        0        0 %
            DSP48A1s                         12       20 %
\end{verbatim}
